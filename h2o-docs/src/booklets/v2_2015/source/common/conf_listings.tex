
%----------------------------------------------------------------------
% Definition for "lstlisting" blocks
%----------------------------------------------------------------------
% --- USAGE ---
%
% \begin{lstlisting}[style=R}
% ...
% \end{lstlisting}
%
% % \begin{lstlisting}[style=output}
% ...
% \end{lstlisting}
%----------------------------------------------------------------------

% By default, make listings all black so it's easy to spot the ones that aren't set to a style.
% This is just a debugging technique.
\lstset{backgroundcolor=\color{black}}

\lstdefinestyle{R}{
  language=R,
  frame=single,
  breaklines,
  basicstyle=\ttfamily,
  commentstyle=\color{gray},% comment style
  numbers=left,% display line numbers on the left side 
  numberstyle=\scriptsize,% use small line numbers 
  numbersep=10pt,% space between line numbers and code
  backgroundcolor=\color{white}
}

\lstdefinestyle{python}{
  language=python,
  frame=single,
  breaklines,
  basicstyle=\ttfamily,
  commentstyle=\color{gray},% comment style
  numbers=left,% display line numbers on the left side 
  numberstyle=\scriptsize,% use small line numbers 
  numbersep=10pt,% space between line numbers and code
  backgroundcolor=\color{white}
}

\definecolor{mygray}{rgb}{0.92,0.92,0.92}

\lstdefinestyle{output}{
  frame=single,
  breaklines,
  basicstyle=\ttfamily,
  numbers=left,% display line numbers on the left side 
  numberstyle=\scriptsize,% use small line numbers 
  numbersep=10pt,% space between line numbers and code
  backgroundcolor=\color{mygray}
}

\newcommand{\waterExampleInR} {
\textbf{Example in R} \\
}

\newcommand{\waterExampleInPython} {
\textbf{Example in Python} \\
}
